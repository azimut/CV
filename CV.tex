\documentclass[11pt]{article}

\usepackage[margin=.5in]{geometry}

\usepackage{hyperref}
\hypersetup{
  colorlinks=true,
  linkcolor=blue,
  urlcolor=blue
}
\urlstyle{same}

\usepackage{titlesec}
\usepackage{titling}
\usepackage{fontawesome}
\renewcommand{\maketitle}{
  \begin{center}{\huge\bfseries\theauthor} \\
    \vspace{.25em}
    {\faEnvelope} \href {mailto:azimut.github@protonmail.com} {azimut.github@protonmail.com}
    ---
    {\faGithub} \href {https://github.com/azimut} {azimut} \\
    \vspace{.25em}
    \today
  \end{center}
}

\usepackage{lastpage}
\usepackage{fancyhdr}
\pagestyle{fancy}
\fancyhf{} % to clear existing header/footer if you don't want it
\cfoot{Page \thepage \hspace{1pt} of \pageref*{LastPage}}

%%%%%%%%%%%%%%%%%%%%%%%%%%%%%%%%%%%%%%%%

\titleformat{\section}
{\huge\bfseries\lowercase}
{}
{0em}
{}[\titlerule]

\titleformat{\subsection}[runin]
{\bfseries}
{}
{0em}
{}

\titleformat{\subsubsection}[runin]
{\normalfont\sffamily} % Formatting of the whole thing
{}          % Numbering
{0em}       % Spacing between number and after
{}          % Anything Between the gap and the title
[]

\titlespacing{\subsubsection}
{0em} % before
{0em} % between
{.5em} % after

%%%%%%%%%%%%%%%%%%%%%%%%%%%%%%%%%%%%%%%%

\begin{document}
\title{R\'esum\'e}
\author{NNNNNN}
\maketitle


\section{Skills}


\begin{tabular}{@{}lll@{}}
  \hspace{.1em} \textbf{Tools} & emacs, vim, svn, git, tcpdump, nmap, docker, ssh, wget, curl, ansible \\
  \hspace{.1em} \textbf{Admin} & Apache, Nginx, Kubernetes, qemu-kvm, dnsmasq, qmail, postfix, dovecot, fluentd \\
  \hspace{.1em} \textbf{DB}    & mysql, mysql percona cluster, postgres, citusdb, elasticsearch, redis, OpenLDAP, Kafka \\
  \hspace{.1em} \textbf{OS}    & Gentoo, Red Hat, ContainerLinux, Debian \\
\end{tabular}


\section{Programming Languages}


\begin{tabular}{@{}lll@{}}
  \hspace{.1em} \textbf{Advanced}            & common lisp, bash \\
  \hspace{.1em} \textbf{Working Experience}  & lua, sql \\
  \hspace{.1em} \textbf{Previous Experience} & golang, erlang, elixir, javascript, python, perl, c, php, glsl, clojure, rust, ocaml \\
\end{tabular}


\section{Idioms}


\begin{tabular}{@{}lll@{}}
  \hspace{.1em} Spanish & Native \\
  \hspace{.1em} English & Upper Intermediate
  \\
\end{tabular}


\section{Job Experience}


\subsection{Email Hosting Software Engineer III --- Web.com \hfill 2017-2018}
\begin{itemize}
  \setlength{\parskip}{0pt}
  \setlength{\itemsep}{0pt plus 1pt}
\item Worked on the conversion of a Qmail based mail platform from Gentoo to Redhat.
\item Created rpms/spec files.
\item Worked on the development of a replacement for Qmail on the platform with a solution based on Dovecot.
\item Implemented a logging solution based on Fluentd, Kafka and Postgres. To identify bad actors and blacklist them accordingly.
\item Worked on optimizing current web performance, adding cache layers and investigating new load balance solutions.
\end{itemize}

\subsection{Web Hosting Software Engineer II --- Web.com \hfill 2014-2016}
\begin{itemize}
  \setlength{\parskip}{0pt}
  \setlength{\itemsep}{0pt plus 1pt}
\item Continuing to work on customer support escalation.
\item Worked on wordpress multinetwork hosting platform, based on nginx and percona cluster (mysql).
\item Packaging and maintaining all the software related to the Gentoo based shared hosting platform.
\item Maintaing the in-house configuration management tools and related PXE servers and custom kernel build.
\item Worked on the development of a nginx based proxy for SSL-termination. Developed in Lua scripts and redis.
\item Deployed a Kubernetes cluster for development of internal applications.
\end{itemize}

\subsection{Web Hosting Software Engineer I --- Web.com \hfill 2011-2013}
\begin{itemize}
  \setlength{\parskip}{0pt}
  \setlength{\itemsep}{0pt plus 1pt}
\item Troubleshooting customer issues on a Unix shared web hosting platform.
\item As the last level of support I was responsable of understanding in depth the issue and fiding the root cause of it.
\item Which consisted of coding issues on the cx code, bad configuration on the server side or missing software configurations.
\end{itemize}

\subsection{Jr. Linux System Administrator --- Sols S.A. \hfill 2011}
\begin{itemize}
  \setlength{\parskip}{0pt}
  \setlength{\itemsep}{0pt plus 1pt}
\item Network design and deploy of a OpenVPN node and squid proxy. Maintenance and technical support.
\end{itemize}


\section{Education}


\begin{tabular}{@{}lll@{}}
  \hspace{.1em} Professional Penetration Tester & Base4 University & 2015 \\
  \hspace{.1em} Red Hat System Administrator I  & IT Collage       & 2014 \\
  \hspace{.1em} Systems Engineer                & UTN - Incomplete & 2010-2011\\
\end{tabular}


\section{Personal Projects}

\subsubsection{\href {https://github.com/azimut/CV} {CV}}
This CV, written in LaTeX, build and publish through Github Actions
\subsubsection{\href {https://github.com/azimut/gantoo} {gantoo}}
A Gentoo Linux Docker container build with Github Actions and deployed to DockerHub. Using musl and oriented to static linking
\subsubsection{\href {https://github.com/azimut/overlay} {overlay}}
A Gentoo Linux overlay, where I wrote build and installation scripts for software not available by default

\subsection{Golang}
\subsubsection{\href {https://github.com/azimut/lainviewer} {lainviewer}}
Json parser and terminal view of forum messages.
%%(\input|"bash loc.sh lainviewer" LOC)
\subsubsection{\href {https://github.com/azimut/redditviewer} {redditviewer}}
Json parser and terminal view of forum messages.
%%(\input|"bash loc.sh redditviewer" LOC)
\subsubsection{\href {https://github.com/azimut/github-rss} {github-rss}}
Using github golang api library to generate RSS feeds.
%%(\input|"bash loc.sh github-rss" LOC)
\subsubsection{\href {https://github.com/azimut/4chan-rss} {4chan-rss}}
Using 4chan golang api library to generate RSS feeds.
%%(\input|"bash loc.sh 4chan-rss" LOC)
\subsubsection{\href {https://github.com/azimut/sunny} {sunny}}
Queries different cloud providers for ranges of IPs they operate and determine if the provided IPs are hosted on them.
%%(\input|"bash loc.sh sunny" LOC)

\subsection{Erlang}
\subsubsection{\href {https://github.com/azimut/snitch} {snitch}}
Monitors public DNS record changes.

\subsection{Rust}
\subsubsection{\href {https://github.com/azimut/ccache} {ccache}}
Saves the compressed output of a command on disk, if it fails, returns the saved output.
%%(\input|"bash loc.sh ccache" LOC)

\subsection{Bash}
\subsubsection{\href {https://github.com/azimut/autoaim} {autoaim}}
Integrate several OSINT tools for automatic recon. Postgresql backed.
%%(\input|"bash loc.sh autoaim" LOC)
\subsubsection{\href {https://github.com/azimut/daily-pic} {daily-pic}}
HTML scrapper. Downloads wallpapers from >10 different sites.
%%(\input|"bash loc.sh daily-pic" LOC)
\subsubsection{\href {https://github.com/azimut/kubler-dock} {kubler-dock}}
Contributor. Using Gentoo's USE feature and docker containers in order to obtain bare minimal docker images.
%%(\input|"bash loc.sh kubler-dock" LOC)

\subsection{Lisp}
\subsubsection{\href {https://github.com/azimut/launchpad-csound} {launchpad-csound}}
MIDI instrument and sequencer for Novation Launchpad.
%%(\input|"bash loc.sh launchpad-csound" LOC)
\subsubsection{\href {https://github.com/azimut/rocketman} {rocketman}}
Client implementation of the TCP based protocol for Rocket Tracker.
%%(\input|"bash loc.sh rocketman" LOC)
\subsubsection{\href {https://github.com/azimut/incandescent} {incandescent}}
3d game engine. Using OpenGL and GLSL for graphics and post-processing effects. OpenAL for positional audio. Open Dynamics Engine for physics. And Common Lisp Object System (CLOS) for game logic.
%%(\input|"bash loc.sh incandescent" LOC)
\subsubsection{\href {https://github.com/azimut/shiny} {shiny}}
Project for on the fly audio and video sequencing (Livecoding). Using different C projects through CFFI calls. Fluidsynth, Supercollider and Csound for music synthesis. And OpenCV and OpenGL for video and shader based animations.
%%(\input|"bash loc.sh shiny" LOC)
\subsubsection{\href {https://github.com/azimut/scenic} {scenic}}
Forward rendering toolkit with support for multiple lights and scenes.
%%(\input|"bash loc.sh scenic" LOC)

\subsection{C}
\subsubsection{\href {https://github.com/azimut/sleeper} {sleeper}}
Small C program using `upower` and `x11` libs to log state changes into systemd.
%%(\input|"bash loc.sh sleeper" LOC)


\end{document}
