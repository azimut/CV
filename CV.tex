\documentclass[11pt]{article}

\usepackage[margin=.5in]{geometry}

\usepackage{hyperref}
\hypersetup{
  colorlinks=true,
  linkcolor=blue,
  urlcolor=blue
}
\urlstyle{same}

\usepackage{multicol}
\setlength{\columnsep}{1cm}
\usepackage{randtext}
\usepackage{titlesec}
\usepackage{titling}
\usepackage{fontawesome5}
\usepackage{xparse}

\usepackage{xcolor}
\usepackage{colortbl}

\usepackage{makecell}
\usepackage{tabularx}
\usepackage{enumitem}
\usepackage{array}
\usepackage{multirow}
\ExplSyntaxOn
\NewDocumentCommand{\getenv}{om}
 {
  \sys_get_shell:nnN { kpsewhich ~ --var-value ~ #2 } { } \l_tmpa_tl
  \tl_trim_spaces:N \l_tmpa_tl
  \IfNoValueTF { #1 }
   {
    \tl_use:N \l_tmpa_tl
   }
   {
    \tl_set_eq:NN #1 \l_tmpa_tl
   }
 }
\ExplSyntaxOff

\getenv[\MAILTO]{MAILTO}
\getenv[\FULLNAME]{FULLNAME}
\getenv[\LINKEDIN]{LINKEDIN}

\renewcommand{\maketitle}{
  \begin{center}{\huge\bfseries\theauthor} \\
    \vspace{.25em}
    {\faEnvelope} \href {mailto:\MAILTO} {\expandafter\randomize\expandafter{\MAILTO}} $\cdot$
    {\faGithub} \href {https://github.com/azimut} {azimut} $\cdot$
    {\faLinkedin} \href {https://www.linkedin.com/in/\LINKEDIN} {\LINKEDIN} \\
    \vspace{.25em}
    \today
  \end{center}
}

\usepackage{lastpage}
\usepackage{fancyhdr}
\renewcommand{\headrulewidth}{0pt}
\pagestyle{fancy}
\fancyhf{} % to clear existing header/footer if you don't want it
\cfoot{Page \thepage \hspace{1pt} of \pageref*{LastPage}}

%%%%%%%%%%%%%%%%%%%%%%%%%%%%%%%%%%%%%%%%

\titleformat{\section}
{\centering\huge\bfseries\lowercase}
{}
{0em}
{}[\titlerule]

\titleformat{\subsection}[runin]
{\bfseries}
{}
{0em}
{}

\titleformat{\subsubsection}[runin]
{\normalfont\sffamily} % Formatting of the whole thing
{}          % Numbering
{0em}       % Spacing between number and after
{}          % Anything Between the gap and the title
[]

\titlespacing{\subsubsection}
{0em} % before
{0em} % between
{.5em} % after

\newcommand{\git}[1]{\href {https://github.com/azimut/#1} {\scriptsize{\faLink}}}
\newcommand
    {\project}
    [3]
    {\texttt{#1} & #3 \git{#2}\\}

\setlist{noitemsep}

%%%%%%%%%%%%%%%%%%%%%%%%%%%%%%%%%%%%%%%%

\begin{document}

\title{R\'esum\'e}
\author{\FULLNAME}

\maketitle


\begin{multicols}{2}
  \section{About Me}
  I spend most of my professional career maintaining and designing high-scalable Linux systems. More recently, I have been learning how those systems are built from the programming side.
  \columnbreak
  \section{Technical Skills}
  \begin{center}
    \begin{tabular}{ c l@{} }
      \raisebox{-0.2\height}{\faCode}     & common lisp, bash, go, awk, jq, elm \\
      \raisebox{-0.2\height}{\faWrench}   & emacs, git, docker, ansible, github actions \\
      \raisebox{-0.2\height}{\faDatabase} & sqlite, postgresql \\
      \raisebox{-0.2\height}{\faLinux}    & gentoo, red hat
    \end{tabular}
  \end{center}
\end{multicols}


\section{Job Experience}


\subsection{Email Hosting Software Engineer III --- Web.com \hfill 2017-2018}
\begin{itemize}
\item Continuing the OS migration now on the shared email platform. From using \texttt{qmail} to \texttt{postfix}.
\item Implemented a centralized logging solution based on the \texttt{Fluentd}, \texttt{Postgresql} and \texttt{Grafana}.
\item Automated \texttt{qemu} vm creation for non-production environments, using ansible.
\item Deployed a \texttt{kubernetes} cluster for running \texttt{selenium} integration tests.
\end{itemize}

\subsection{Web Hosting Software Engineer II --- Web.com \hfill 2014-2016}
\begin{itemize}
\item Development on a dynamic ssl web proxy, based on \texttt{nginx} and \texttt{lua}. Backed by \texttt{redis}.
\item Troubleshooting issues on a multi-network \texttt{Wordpress} installation. Backed by \texttt{mysql} Percona Cluster.
\item Single-handedly migrated part of the web hosting platform from Gentoo to RedHat using \texttt{Ansible}.
\end{itemize}

\subsection{Web Hosting Software Engineer I --- Web.com \hfill 2011-2013}
\begin{itemize}
\item Troubleshooting L3 customer support requests on a \texttt{LAMP} based shared hosting platform.
\item OS maintainer of a Gentoo Linux fork used across the shared hosting platform. From kernel to libraries.
\end{itemize}

\subsection{Jr. Linux System Administrator --- Sols S.A. \hfill 2011}
\begin{itemize}
\item Network design and deploy of a \texttt{OpenVPN} node and \texttt{Squid} proxy. Maintenance and technical support.
\end{itemize}

\begin{multicols}{2}
  \section{Education}
    \begin{tabular}{ r | p{0.19\textwidth} | l }
      15    & Penetration Tester  & Base4 \\
      14    & RHSA I              & IT Collage \\
      10/11 & Systems Engineering & UTN - Incomplete
    \end{tabular}
  \columnbreak
  \section{Languages}
  \begin{tabular}{ l l }
    \hspace{.1em} \textbf{English} & Upper Intermediate \\
    \hspace{.1em} \textbf{Spanish} & Native \\
  \end{tabular}
\end{multicols}


\section{Personal Projects}


\hypersetup{urlcolor=gray}
\begin{center}
  \begin{tabular}{ c l }
    \project{Common Lisp} {scenic}         {several game engines and midi sequencers}
    \project{Golang}      {cli-view}       {command line clients to view comments of different websites}
    \project{Elm}         {newspod}        {fully searchable database of tech podcasts}
    \project{Erlang}      {snitch}         {monitor of public dns records}
    \project{Typescript}  {reddit-gallery} {reddit image gallery made with React}
  \end{tabular}
\end{center}

\end{document}
