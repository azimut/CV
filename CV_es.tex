\documentclass[11pt]{article}

\usepackage[spanish,english]{babel}

\usepackage[
  left   = 0.5in,
  right  = 0.5in,
  top    = 0.35in,
  bottom = 0.5in
]{geometry}

\usepackage{hyperref}
\hypersetup{
  colorlinks = true,
  linkcolor  = blue,
  urlcolor   = blue
}
\urlstyle{same}

\usepackage{qrcode}
\usepackage{setspace}
\usepackage[spanish,english]{babel}
\usepackage{multicol}
\setlength{\columnsep}{1cm}
\usepackage{randtext}
\usepackage{titlesec}
\usepackage{titling}
\usepackage{fontawesome5}
\usepackage{xcolor}
\usepackage{enumitem}

\ExplSyntaxOn
\NewDocumentCommand{\getenv}{om}
 {
  \sys_get_shell:nnN { kpsewhich ~ --var-value ~ #2 } { } \l_tmpa_tl
  \tl_trim_spaces:N \l_tmpa_tl
  \IfNoValueTF { #1 }
   {
    \tl_use:N \l_tmpa_tl
   }
   {
    \tl_set_eq:NN #1 \l_tmpa_tl
   }
 }
\ExplSyntaxOff

\getenv[\MAILTO]{MAILTO}
\getenv[\FULLNAME]{FULLNAME}
\getenv[\LINKEDIN]{LINKEDIN}

\newenvironment{nscenter}
 {\parskip=0pt\par\nopagebreak\centering}
 {\par\noindent\ignorespacesafterend}

\renewcommand{\maketitle}[2]{
  \begin{nscenter}
    \begin{tabular}{ r | c | r }
      \qrcode{https://azimut.github.io/CV/}
      &
      \begin{tabular}{ c }
        {\Huge\bfseries\theauthor} \\
        #1
      \end{tabular}
      &
      \setlength{\tabcolsep}{3pt}
      \begin{tabular}{ c l }
        \raisebox{-0.10\height}{\small{\faEnvelope}} & \href {mailto:\MAILTO}                        {\expandafter\randomize\expandafter{\MAILTO}} \\
        \raisebox{-0.05\height}{\small{\faGithub}}   & \href {https://github.com/azimut}             {azimut} \\
        \raisebox{-0.05\height}{\small{\faLinkedin}} & \href {https://www.linkedin.com/in/\LINKEDIN} {\LINKEDIN} \\
        \raisebox{-0.05\height}{\small{\faHome}}     & \href {https://azimut.github.io/}             {#2}
      \end{tabular}
    \end{tabular}
    \vspace{.1in}
  \end{nscenter}
}

\usepackage{lastpage}
\usepackage{fancyhdr}
\renewcommand{\headrulewidth}{0pt}
\pagestyle{fancy}
\fancyhf{} % to clear existing header/footer if you don't want it
\cfoot{\iflanguage{spanish}{Página}{Page} \thepage \hspace{1pt} \iflanguage{spanish}{de}{of} \pageref*{LastPage}}

%%%%%%%%%%%%%%%%%%%%%%%%%%%%%%%%%%%%%%%%

\titleformat{\section}
{\centering\huge\bfseries\lowercase}
{}
{0em}
{}[\titlerule]

\titleformat{\subsection}[runin]
{\bfseries}
{}
{0em}
{}

\titleformat{\subsubsection}[runin]
{\normalfont\sffamily} % Formatting of the whole thing
{}          % Numbering
{0em}       % Spacing between number and after
{}          % Anything Between the gap and the title
[]

\titlespacing{\subsubsection}
{0em} % before
{0em} % between
{.5em} % after

\newcommand{\git}[1]{\href {https://github.com/azimut/#1} {\scriptsize{\faLink}}}
\newcommand
    {\project}
    [3]
    {\texttt{#1} & #3 \git{#2}\\}

\newcommand
    {\skill}
    [2]
    {\raisebox{-0.15\height}{\footnotesize{#1}} & #2 \\}

\setlist{noitemsep}

\newcommand
    {\makeabout}
    [1]
    {\begin{multicols}{2}
        \section{\iflanguage{spanish}{Acerca de M\'i}{About Me}}
        #1
        \columnbreak
        \section{\iflanguage{spanish}{Conocimientos T\'ecnicos}{Technical Skills}}
        \begin{nscenter}
          \setlength{\tabcolsep}{3pt}
          \begin{tabular}{ c l@{} }
            \skill {\faCode}     {common lisp, bash, go, awk, jq, elm}
            \skill {\faWrench}   {emacs, github actions, makefile, docker, ansible}
            \skill {\faDatabase} {sqlite, postgresql}
            \skill {\faLinux}    {gentoo, red hat}
          \end{tabular}
        \end{nscenter}
    \end{multicols}}

\newcommand
    {\makeeducation}
    [6]
    {\begin{multicols}{2}
        \section{#1}
        \begin{tabular}{ r | p{0.2\textwidth} | l }
          24    & Front-End JS        & BA "Aprende" \\
          15    & Penetration Tester  & Base4 \\
          14    & RHSA I              & IT Collage \\
          10/11 & Systems Engineering & UTN - Incomplete
        \end{tabular}
        \columnbreak
        \section{#2}
        \begin{tabular}{ l l }
          \hspace{.1em} \textbf{#3} & #4 \\
          \hspace{.1em} \textbf{#5} & #6
        \end{tabular}
    \end{multicols}}


\cfoot{Página 1 de 1}

\begin{document}

\title{R\'esum\'e}
\author{\FULLNAME}

\maketitle{\selectlanguage{spanish}\today}{sitio personal}

\selectlanguage{spanish}{
  \makeabout{
    Dedique mi carrera profesional a mantener sistemas Linux altamente escalables.
    Recientemente, he estado aprendiendo como son esos sistemas construidos desde el lado programacional.
  }
}

\section{Experiencia Laboral}


\subsection{Email Hosting Software Engineer III --- Web.com \hfill 2017-2018}
\begin{itemize}
\item Continuacion de la migracion de sistema operativo, usando ansible, en la plataforma de mail.
\item Implemente una plataforma de logging centralizado basado en \texttt{Fluentd}, \texttt{Postgresql} y \texttt{Grafana}.
\item Automatice la creacion e imaging de vms en qemu para entornos no productivos, usando ansible.
\end{itemize}

\subsection{Web Hosting Software Engineer II --- Web.com \hfill 2014-2016}
\begin{itemize}
\item Desarrollo en una plataforma de proxy dinamico basado en \texttt{nginx} y \texttt{Lua}. Con backend end \texttt{redis}.
\item Mantenimiento en una plataforma dedicada al hosting multi-network de \texttt{Wordpress}. Con backend en \texttt{Percona Cluster}
\item Responsable de la migracion a Red Hat, para la plataforma de hosting compartido, usando \texttt{Ansible}.

\end{itemize}

\subsection{Web Hosting Software Engineer I --- Web.com \hfill 2011-2013}
\begin{itemize}
\item Resolucion de problemas y soporte tecnico L3 en un entorno \texttt{LAMP} para hosting web.
\item Mantenimiento de un fork del sistema operativo Gentoo Linux, usado en la plataforma de hosting compartido.
\end{itemize}

\subsection{Jr. Linux System Administrator --- Sols S.A. \hfill 2011}
\begin{itemize}
\item Soporte tecnico y mantenimiento de una red interna privada usando \texttt{OpenVPN} y un proxy \texttt{Squid}.
\end{itemize}


\selectlanguage{spanish}{
  \makeeducation
}


\section{Proyectos Personales}

\hypersetup{
  pdftitle = {{\FULLNAME}},
  urlcolor = gray
}
\begin{center}
  \begin{tabular}{ c l }
    \project{Common Lisp} {scenic}         {varios motores gr\'aficos, y secuenciadores de MIDI}
    \project{Golang}      {cli-view}       {clientes de terminal, para ver los comentarios de diferentes sitios}
    \project{Elm}         {newspod}        {motor de b\'usqueda online de podcasts}
    \project{C}           {sleeper}        {monitoreo y almacenamiento en sqlite de eventos X11}
  \end{tabular}
\end{center}

\end{document}
